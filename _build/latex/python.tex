%% Generated by Sphinx.
\def\sphinxdocclass{jupyterBook}
\documentclass[letterpaper,10pt,english]{jupyterBook}
\ifdefined\pdfpxdimen
   \let\sphinxpxdimen\pdfpxdimen\else\newdimen\sphinxpxdimen
\fi \sphinxpxdimen=.75bp\relax
\ifdefined\pdfimageresolution
    \pdfimageresolution= \numexpr \dimexpr1in\relax/\sphinxpxdimen\relax
\fi
%% let collapsible pdf bookmarks panel have high depth per default
\PassOptionsToPackage{bookmarksdepth=5}{hyperref}
%% turn off hyperref patch of \index as sphinx.xdy xindy module takes care of
%% suitable \hyperpage mark-up, working around hyperref-xindy incompatibility
\PassOptionsToPackage{hyperindex=false}{hyperref}
%% memoir class requires extra handling
\makeatletter\@ifclassloaded{memoir}
{\ifdefined\memhyperindexfalse\memhyperindexfalse\fi}{}\makeatother

\PassOptionsToPackage{warn}{textcomp}

\catcode`^^^^00a0\active\protected\def^^^^00a0{\leavevmode\nobreak\ }
\usepackage{cmap}
\usepackage{fontspec}
\defaultfontfeatures[\rmfamily,\sffamily,\ttfamily]{}
\usepackage{amsmath,amssymb,amstext}
\usepackage{polyglossia}
\setmainlanguage{english}



\setmainfont{FreeSerif}[
  Extension      = .otf,
  UprightFont    = *,
  ItalicFont     = *Italic,
  BoldFont       = *Bold,
  BoldItalicFont = *BoldItalic
]
\setsansfont{FreeSans}[
  Extension      = .otf,
  UprightFont    = *,
  ItalicFont     = *Oblique,
  BoldFont       = *Bold,
  BoldItalicFont = *BoldOblique,
]
\setmonofont{FreeMono}[
  Extension      = .otf,
  UprightFont    = *,
  ItalicFont     = *Oblique,
  BoldFont       = *Bold,
  BoldItalicFont = *BoldOblique,
]



\usepackage[Bjarne]{fncychap}
\usepackage[,numfigreset=1,mathnumfig]{sphinx}

\fvset{fontsize=\small}
\usepackage{geometry}


% Include hyperref last.
\usepackage{hyperref}
% Fix anchor placement for figures with captions.
\usepackage{hypcap}% it must be loaded after hyperref.
% Set up styles of URL: it should be placed after hyperref.
\urlstyle{same}

\addto\captionsenglish{\renewcommand{\contentsname}{Introduction}}

\usepackage{sphinxmessages}



        % Start of preamble defined in sphinx-jupyterbook-latex %
         \usepackage[Latin,Greek]{ucharclasses}
        \usepackage{unicode-math}
        % fixing title of the toc
        \addto\captionsenglish{\renewcommand{\contentsname}{Contents}}
        \hypersetup{
            pdfencoding=auto,
            psdextra
        }
        % End of preamble defined in sphinx-jupyterbook-latex %
        

\title{Battery Application}
\date{Apr 19, 2022}
\release{}
\author{FEV Documentation Team}
\newcommand{\sphinxlogo}{\vbox{}}
\renewcommand{\releasename}{}
\makeindex
\begin{document}

\pagestyle{empty}
\sphinxmaketitle
\pagestyle{plain}
\sphinxtableofcontents
\pagestyle{normal}
\phantomsection\label{\detokenize{index::doc}}


\sphinxAtStartPar
Welcome to \sphinxstylestrong{SCALE} Battery Documentation!
\begin{quote}

\sphinxAtStartPar
\sphinxstyleemphasis{Essential element of new transportation area, batteries are key part of the energy transition. In constant evolution, battery testing has to be innovative and reactive!}
\end{quote}

\sphinxAtStartPar
SCALE Battery (\sphinxstylestrong{S}tandard and \sphinxstylestrong{C}onfigurable \sphinxstylestrong{A}pplication for \sphinxstylestrong{L}aboratory \sphinxstylestrong{E}nvirronment) is a MORPHEE® application for battery test cells. It allows to perform test on:
\begin{itemize}
\item {} 
\sphinxAtStartPar
Cells,

\item {} 
\sphinxAtStartPar
Module

\item {} 
\sphinxAtStartPar
Pack.

\end{itemize}

\begin{sphinxadmonition}{note}{Note:}
\sphinxAtStartPar
\sphinxstylestrong{SCALE} also exists for many different kind of test cell on automotive market:

\begin{sphinxuseclass}{sphinx-bs}
\begin{sphinxuseclass}{container}
\begin{sphinxuseclass}{pb-4}
\begin{sphinxuseclass}{row}
\begin{sphinxuseclass}{d-flex}
\begin{sphinxuseclass}{col-lg-6}
\begin{sphinxuseclass}{col-md-6}
\begin{sphinxuseclass}{col-sm-6}
\begin{sphinxuseclass}{col-xs-12}
\begin{sphinxuseclass}{p-2}
\begin{sphinxuseclass}{card}
\begin{sphinxuseclass}{w-100}
\begin{sphinxuseclass}{shadow}
\begin{sphinxuseclass}{card-body}\begin{itemize}
\item {} 
\sphinxAtStartPar
\sphinxstylestrong{SCALE Component} : A basic to test all your automotive components.

\item {} 
\sphinxAtStartPar
\sphinxstylestrong{SCALE Engine} : The state of the art for all of your thermal and hybrid engine test benches. From End\sphinxhyphen{}Of\sphinxhyphen{}Line to Research \& Development solution.

\item {} 
\sphinxAtStartPar
\sphinxstylestrong{SCALE Emotor} : Adapt your test benches and working methods to new market trends. E\sphinxhyphen{}mobility is more than ever a reality!

\end{itemize}

\end{sphinxuseclass}
\end{sphinxuseclass}
\end{sphinxuseclass}
\end{sphinxuseclass}
\end{sphinxuseclass}
\end{sphinxuseclass}
\end{sphinxuseclass}
\end{sphinxuseclass}
\end{sphinxuseclass}
\end{sphinxuseclass}
\begin{sphinxuseclass}{d-flex}
\begin{sphinxuseclass}{col-lg-6}
\begin{sphinxuseclass}{col-md-6}
\begin{sphinxuseclass}{col-sm-6}
\begin{sphinxuseclass}{col-xs-12}
\begin{sphinxuseclass}{p-2}
\begin{sphinxuseclass}{card}
\begin{sphinxuseclass}{w-100}
\begin{sphinxuseclass}{shadow}
\begin{sphinxuseclass}{card-body}\begin{itemize}
\item {} 
\sphinxAtStartPar
\sphinxstylestrong{SCALE Fuel Cell} : The new trend for power generation in modern vehicles. Tomorrow’s ideas have already a SCALE solution today.

\item {} 
\sphinxAtStartPar
\sphinxstylestrong{SCALE Epowertrain/Powertrain} : Testing the complete powertrain in its conventional, hybrid or electrical configuration is now possible.

\item {} 
\sphinxAtStartPar
\sphinxstylestrong{SCALE Vehicle} : For Vehicle on chassis dynamometer testing

\end{itemize}

\end{sphinxuseclass}
\end{sphinxuseclass}
\end{sphinxuseclass}
\end{sphinxuseclass}
\end{sphinxuseclass}
\end{sphinxuseclass}
\end{sphinxuseclass}
\end{sphinxuseclass}
\end{sphinxuseclass}
\end{sphinxuseclass}
\end{sphinxuseclass}
\end{sphinxuseclass}
\end{sphinxuseclass}
\end{sphinxuseclass}\end{sphinxadmonition}


\sphinxstrong{See also:}
\nopagebreak


\sphinxAtStartPar
Get more info on \sphinxhref{https://morphee.fev.com}{\sphinxstylestrong{MORPHEE}},




\part{Introduction}


\chapter{General presentation}
\label{\detokenize{01_General-presentation:general-presentation}}\label{\detokenize{01_General-presentation::doc}}

\section{Introduction}
\label{\detokenize{01_General-presentation:introduction}}
\sphinxAtStartPar
\sphinxstylestrong{SCALE Battery} has been developed to address all type of battery;
\begin{itemize}
\item {} 
\sphinxAtStartPar
Cell

\item {} 
\sphinxAtStartPar
Module

\item {} 
\sphinxAtStartPar
Pack

\end{itemize}

\sphinxAtStartPar
To meet the market demand, we are using the \DUrole{xref,myst}{MORPHEE® multi\sphinxhyphen{}instance} concept.
This allows to start on the the same computer;
\begin{itemize}
\item {} 
\sphinxAtStartPar
Up to 32 cell tests in parallel

\item {} 
\sphinxAtStartPar
Up to 16 Module tests in parallel

\item {} 
\sphinxAtStartPar
Up to 8 Packs tests in parallel

\end{itemize}


\section{Battery Testing principle}
\label{\detokenize{01_General-presentation:battery-testing-principle}}

\subsection{Equipements and measure}
\label{\detokenize{01_General-presentation:equipements-and-measure}}
\sphinxAtStartPar
To test a battery we need to have:
\begin{itemize}
\item {} 
\sphinxAtStartPar
\sphinxstylestrong{An Energy System} >> Allow to charge and discharge the battery. Generally it also manage the re\sphinxhyphen{}injection on the network.

\item {} 
\sphinxAtStartPar
\sphinxstylestrong{A climatic chamber} \sphinxhyphen{} \sphinxstyleemphasis{optional} >> Allow to simulate environmental conditions

\item {} 
\sphinxAtStartPar
\sphinxstylestrong{A coolant conditionning system} \sphinxhyphen{} \sphinxstyleemphasis{optional} >> use to cooldown the battery in same condition as in the vehicle.

\item {} 
\sphinxAtStartPar
\sphinxstylestrong{A BMS} (\sphinxstylestrong{B}attery \sphinxstylestrong{M}angement \sphinxstylestrong{S}ystem) \sphinxhyphen{} \sphinxstyleemphasis{optional} >> Battery control unit handling the different cells/modules/pack and different modes (charge/sleep/Drive)

\item {} 
\sphinxAtStartPar
\sphinxstyleemphasis{\sphinxstylestrong{Additionnal acquisition}} \sphinxhyphen{} \sphinxstyleemphasis{optional} >> Allow to add voltage, current and temperature measurement.

\end{itemize}

\sphinxAtStartPar
\sphinxincludegraphics{{Battery_test_concept}.png}

\begin{sphinxadmonition}{note}{Note:}
\sphinxAtStartPar
Except Energy System, all others equipement can be ‘Optional’, depending on each application.
\end{sphinxadmonition}


\subsection{Energy system channel concept}
\label{\detokenize{01_General-presentation:energy-system-channel-concept}}
\sphinxAtStartPar
Energy System have one or several channel allowing to address one or several battery:
\begin{itemize}
\item {} 
\sphinxAtStartPar
If there is a need for more current, channels can we wired in parallel

\item {} 
\sphinxAtStartPar
If there is a need for more voltage, channels can be wired in serial

\end{itemize}


\begin{savenotes}\sphinxattablestart
\centering
\begin{tabulary}{\linewidth}[t]{|T|T|}
\hline
\sphinxstyletheadfamily 
\sphinxAtStartPar
Parallel connexion
&\sphinxstyletheadfamily 
\sphinxAtStartPar
Serial connexion
\\
\hline
\sphinxAtStartPar
\sphinxincludegraphics{{Battery_test_parallel}.png}
&
\sphinxAtStartPar
\sphinxincludegraphics{{Battery_test_serial}.png}
\\
\hline
\end{tabulary}
\par
\sphinxattableend\end{savenotes}

\sphinxAtStartPar
Regarding equipement capability, it is also possible to combine Energy system cascading and channel parallelisation or serialisation.


\begin{savenotes}\sphinxattablestart
\centering
\begin{tabulary}{\linewidth}[t]{|T|}
\hline
\sphinxstyletheadfamily 
\sphinxAtStartPar
Equipement and channel cascading
\\
\hline
\sphinxAtStartPar
\sphinxincludegraphics{{Battery_test_mixed}.png}
\\
\hline
\end{tabulary}
\par
\sphinxattableend\end{savenotes}


\subsection{Modularity}
\label{\detokenize{01_General-presentation:modularity}}
\sphinxAtStartPar
Generally, in one climatic chamber there will be several battery in test. Climatic condition are identical but tests can be different!


\begin{savenotes}\sphinxattablestart
\centering
\begin{tabulary}{\linewidth}[t]{|T|T|T|}
\hline
\sphinxstyletheadfamily 
\sphinxAtStartPar
1 battery
&\sphinxstyletheadfamily 
\sphinxAtStartPar
4 batteries
&\sphinxstyletheadfamily 
\sphinxAtStartPar
12 batteries
\\
\hline
\sphinxAtStartPar
\sphinxincludegraphics{{Battery-modulo-1}.png}
&
\sphinxAtStartPar
\sphinxincludegraphics{{Battery-modulo-4}.png}
&
\sphinxAtStartPar
\sphinxincludegraphics{{Battery-Modulo-12}.png}
\\
\hline
\end{tabulary}
\par
\sphinxattableend\end{savenotes}

\begin{sphinxadmonition}{important}{Important:}
\sphinxAtStartPar
The quantity of test is directly linked to the quantity of Energy System channels.
For example, if globaly a system has 8 channels, we can:
\begin{itemize}
\item {} 
\sphinxAtStartPar
Start 8 independant test using 1 channel each

\item {} 
\sphinxAtStartPar
Start 4 independant test using 2 channel each (in // or serial)

\item {} 
\sphinxAtStartPar
Start 2 independant test using 4 channel each (in // or serial)

\end{itemize}

\sphinxAtStartPar
Possibility is given, in one side from the hardware, and in the other side from the software configuration.
\end{sphinxadmonition}


\section{Software architecture}
\label{\detokenize{01_General-presentation:software-architecture}}
\sphinxAtStartPar
Software architecture is based in one hand, on a master application managing most of the equipments (Energy, cooling, acquisition, climatic…), and on another hand several procedure executing the test(call instances). Communication between main application and procedure is developped the in {\hyperref[\detokenize{02_Multi-Instance::doc}]{\sphinxcrossref{\DUrole{doc,std,std-doc}{MORPHEE® multi\sphinxhyphen{}instance}}}} chapter.

\sphinxAtStartPar
\sphinxincludegraphics{{Battery_Concept}.png}

\sphinxAtStartPar
Objective is to keep all instances as identical as possible. Only some interfaces can change;
\begin{itemize}
\item {} 
\sphinxAtStartPar
BMS

\item {} 
\sphinxAtStartPar
Specif devices.

\end{itemize}

\sphinxAtStartPar
A specific configuration (hardware connexion) can also be done for every instance..


\section{Instances capability}
\label{\detokenize{01_General-presentation:instances-capability}}
\sphinxAtStartPar
We limit the quantities of instances to 1 master + 32 instances.
But more generally, the quantities will be limited by the configuration you will select:


\begin{savenotes}\sphinxattablestart
\centering
\begin{tabulary}{\linewidth}[t]{|T|T|}
\hline
\sphinxstyletheadfamily 
\sphinxAtStartPar
Ref
&\sphinxstyletheadfamily 
\sphinxAtStartPar
Description
\\
\hline
\sphinxAtStartPar
\sphinxstylestrong{SA\sphinxhyphen{}M64\sphinxhyphen{}MULTI\sphinxhyphen{}04}
&
\sphinxAtStartPar
Possibility to start \sphinxstylestrong{1} Master Instance + \sphinxstylestrong{4} test instances
\\
\hline
\sphinxAtStartPar
\sphinxstylestrong{SA\sphinxhyphen{}M64\sphinxhyphen{}MULTI\sphinxhyphen{}08}
&
\sphinxAtStartPar
Possibility to start \sphinxstylestrong{1} Master Instance + \sphinxstylestrong{8} test instances
\\
\hline
\sphinxAtStartPar
\sphinxstylestrong{SA\sphinxhyphen{}M64\sphinxhyphen{}MULTI\sphinxhyphen{}16}
&
\sphinxAtStartPar
Possibility to start \sphinxstylestrong{1} Master Instance + \sphinxstylestrong{16} test instances
\\
\hline
\sphinxAtStartPar
\sphinxstylestrong{SA\sphinxhyphen{}M64\sphinxhyphen{}MULTI\sphinxhyphen{}24}
&
\sphinxAtStartPar
Possibility to start \sphinxstylestrong{1} Master Instance + \sphinxstylestrong{24} test instances
\\
\hline
\sphinxAtStartPar
\sphinxstylestrong{SA\sphinxhyphen{}M64\sphinxhyphen{}MULTI\sphinxhyphen{}32}
&
\sphinxAtStartPar
Possibility to start \sphinxstylestrong{1} Master Instance + \sphinxstylestrong{32} test instances
\\
\hline
\end{tabulary}
\par
\sphinxattableend\end{savenotes}


\section{Instances limitation}
\label{\detokenize{01_General-presentation:instances-limitation}}
\sphinxAtStartPar
Actual limitations are as follow:


\begin{savenotes}\sphinxattablestart
\centering
\begin{tabulary}{\linewidth}[t]{|T|T|T|}
\hline
\sphinxstyletheadfamily 
\sphinxAtStartPar
Ref
&\sphinxstyletheadfamily 
\sphinxAtStartPar
Master
&\sphinxstyletheadfamily 
\sphinxAtStartPar
Instances
\\
\hline
\sphinxAtStartPar
\sphinxstylestrong{Up to 1 + 8 instances}
&
\sphinxAtStartPar
1 kHz
&
\sphinxAtStartPar
1 kHz
\\
\hline
\sphinxAtStartPar
\sphinxstylestrong{Up to 1 + 16 instances}
&
\sphinxAtStartPar
500 Hz
&
\sphinxAtStartPar
500 Hz
\\
\hline
\sphinxAtStartPar
\sphinxstylestrong{Up to 1 + 32 instances}
&
\sphinxAtStartPar
100 Hz
&
\sphinxAtStartPar
100 Hz
\\
\hline
\end{tabulary}
\par
\sphinxattableend\end{savenotes}

\begin{sphinxadmonition}{note}{Note:}
\sphinxAtStartPar
This limitations has been achieved with FEV 8 cores computer.
\end{sphinxadmonition}


\chapter{MORPHEE® Multi\sphinxhyphen{}instance}
\label{\detokenize{02_Multi-Instance:morphee-multi-instance}}\label{\detokenize{02_Multi-Instance::doc}}
\sphinxAtStartPar
Born from our internal need to have independent application running on the same hardware architecture, \sphinxstylestrong{MORPHEE®} multi\sphinxhyphen{}instance has been developed to ensure multiple tests execution independently in terms of procedure and securities.


\section{General concept}
\label{\detokenize{02_Multi-Instance:general-concept}}
\sphinxAtStartPar
In order to meet the growing need for multi\sphinxhyphen{}UUT testing, especially in the field of battery, and after many feasibilities, \sphinxstylestrong{FEV} has developed a new \sphinxstylestrong{MORPHEE®} multi\sphinxhyphen{}instance concept.

\sphinxAtStartPar
Based on this, it is now possible to run several \sphinxstylestrong{MORPHEE®} on the same computer.  The first instance is seen as the ‘master instance’ (or main instance). It can be used to orchestrate all the other instances, to centralize hardware, share channels and distribute events.

\sphinxAtStartPar
All others instances are completely independent and can perform test in real time using own channels, screens, alarms, methods, etc..

\sphinxAtStartPar
\sphinxincludegraphics{{Multi-Instance-Concept}.png}


\section{Multi\sphinxhyphen{}Instance activation and configuration}
\label{\detokenize{02_Multi-Instance:multi-instance-activation-and-configuration}}

\subsection{Activation}
\label{\detokenize{02_Multi-Instance:activation}}
\sphinxAtStartPar
Activation of the multi\sphinxhyphen{}instance can be done quickly, as it is sufficient to specify in \sphinxstylestrong{MENV} (\sphinxstylestrong{M}ORPHEE \sphinxstylestrong{Env}ironment software) that an additional instance is required.

\sphinxAtStartPar
You can add as many instances as needed, and also delete them if they are not needed anymore;

\sphinxAtStartPar
\sphinxincludegraphics{{Multi-Instance-Activation}.png}

\sphinxAtStartPar
If no folders exist for multi\sphinxhyphen{}instance, \sphinxstylestrong{MENV} will create them and populate then with files from Master instance.

\sphinxAtStartPar
\sphinxincludegraphics{{Multi-Instance-Folder}.png}

\sphinxAtStartPar
If folders are already existing, no actions will be performed.


\subsection{Configuration}
\label{\detokenize{02_Multi-Instance:configuration}}
\sphinxAtStartPar
However, activation is not enough, as the whole system has to be configured in order to be able to run a test.
Configuration of each instance is done directly from \sphinxstylestrong{MENV} layer;

\sphinxAtStartPar
\sphinxincludegraphics{{Multi-Instance-Config}.png}

\sphinxAtStartPar
Not all parameters have to be modified as we fixed some rules to simplify global configuration. Only:
\begin{itemize}
\item {} 
\sphinxAtStartPar
Initialization file (\sphinxcode{\sphinxupquote{Morphee.ini}})

\item {} 
\sphinxAtStartPar
Physical Configuration file (\sphinxcode{\sphinxupquote{Morphee.cfp}})

\item {} 
\sphinxAtStartPar
Editor configuration (\sphinxcode{\sphinxupquote{UEditor.config}})

\item {} 
\sphinxAtStartPar
Test Path and Result directory

\end{itemize}

\sphinxAtStartPar
All other parameters are common to all instances, mainly:
\begin{itemize}
\item {} 
\sphinxAtStartPar
Components

\item {} 
\sphinxAtStartPar
Templates

\item {} 
\sphinxAtStartPar
Themes

\item {} 
\sphinxAtStartPar
Skins

\item {} 
\sphinxAtStartPar
Reports

\item {} 
\sphinxAtStartPar
Automatic Session

\item {} 
\sphinxAtStartPar
Binaries•	…

\end{itemize}


\section{Working with Instances}
\label{\detokenize{02_Multi-Instance:working-with-instances}}

\subsection{Editing / Starting a specific instance}
\label{\detokenize{02_Multi-Instance:editing-starting-a-specific-instance}}
\sphinxAtStartPar
By using MToolBar, you will every time prompted to know which instance you want to use;

\sphinxAtStartPar
\sphinxincludegraphics{{Multi-Instance-Mtoolbar}.png}

\sphinxAtStartPar
For;
\begin{itemize}
\item {} 
\sphinxAtStartPar
Starting \sphinxstylestrong{MORPHEE® Editor}

\item {} 
\sphinxAtStartPar
Starting \sphinxstylestrong{MORPHEE® Executive}

\end{itemize}

\sphinxAtStartPar
Information of instance number is displayed on main Form and also in the main toolbar: \sphinxincludegraphics{{Multi-Instance-numbering}.png}

\sphinxAtStartPar
It can be very complex to work with many instances at the same time. Some basic rules can help to avoid problem link to quantity of place where modification can be done.

\sphinxAtStartPar
For example:
\begin{itemize}
\item {} 
\sphinxAtStartPar
Use Meditor instance, only for configuration:
\begin{itemize}
\item {} 
\sphinxAtStartPar
Adding and configuring features

\item {} 
\sphinxAtStartPar
Hardware links

\item {} 
\sphinxAtStartPar
Test cell configuration

\end{itemize}

\item {} 
\sphinxAtStartPar
Use Meditor master instance for all others things:
\begin{itemize}
\item {} 
\sphinxAtStartPar
Master instance configuration

\item {} 
\sphinxAtStartPar
Component writing

\item {} 
\sphinxAtStartPar
Test Writing

\item {} 
\sphinxAtStartPar
Channel calibration

\item {} 
\sphinxAtStartPar
Etc…

\end{itemize}

\end{itemize}


\begin{savenotes}\sphinxattablestart
\centering
\begin{tabulary}{\linewidth}[t]{|T|T|}
\hline
\sphinxstyletheadfamily 
\sphinxAtStartPar
Master Instance
&\sphinxstyletheadfamily 
\sphinxAtStartPar
Others instances
\\
\hline
\sphinxAtStartPar
\sphinxincludegraphics{{Multi-Instance-Master}.png}
&
\sphinxAtStartPar
\sphinxincludegraphics{{Multi-Instance-Instance}.png}
\\
\hline
\end{tabulary}
\par
\sphinxattableend\end{savenotes}


\subsection{Exchanging data between instances}
\label{\detokenize{02_Multi-Instance:exchanging-data-between-instances}}
\sphinxAtStartPar
Each instance can be completely autonomous. Nevertheless, in most cases, it is necessary to share measurement channels, or simply status paths, between the instances.
To make instance independent and use same component, following best practices could be used:

\sphinxAtStartPar
\sphinxstyleemphasis{\sphinxstyleemphasis{\sphinxstylestrong{Sharing a channel from master instance to all instances}}}

\sphinxAtStartPar
The name of the channel in master instance has to be the same in others instance. The channel type has to be \sphinxcode{\sphinxupquote{{[}SHARED\_IN{]}}}

\sphinxAtStartPar
\sphinxincludegraphics{{Multi-Instance-SharedAll}.png}

\sphinxAtStartPar
\sphinxstyleemphasis{\sphinxstyleemphasis{\sphinxstylestrong{Sharing a channel from master to a specific instance}}}

\sphinxAtStartPar
To keep instance generic and use the same channel name, the physical channel has to redirect the standard name from the instance.

\sphinxAtStartPar
\sphinxincludegraphics{{Multi-Instance-SharedIn}.png}

\sphinxAtStartPar
\sphinxstyleemphasis{\sphinxstyleemphasis{\sphinxstylestrong{Getting channel from instances}}}

\sphinxAtStartPar
To send value of channel to master, it is also possible to use the same principle, but channel type must be \sphinxcode{\sphinxupquote{{[}SHARED\sphinxhyphen{}OUT{]}}}.

\sphinxAtStartPar
\sphinxincludegraphics{{Multi-Instance-Sharedout}.png}


\part{Configuration}


\chapter{First Installation}
\label{\detokenize{03_First-Installation:first-installation}}\label{\detokenize{03_First-Installation::doc}}

\section{Software Installation}
\label{\detokenize{03_First-Installation:software-installation}}
\sphinxAtStartPar
Use of MORPHEE® 3.4.x

\sphinxAtStartPar
Use of last SCALE Battery version

\sphinxAtStartPar
All version and setup can be download here >> \sphinxurl{https://download.fev-software.com/DownloadPortal/pages/DownloadPortal}


\section{Licensing / Protection}
\label{\detokenize{03_First-Installation:licensing-protection}}
\sphinxAtStartPar
Get protection codes:
\begin{itemize}
\item {} 
\sphinxAtStartPar
SCALE Battery
\begin{itemize}
\item {} 
\sphinxAtStartPar
Management

\item {} 
\sphinxAtStartPar
Control management.BATTERY

\item {} 
\sphinxAtStartPar
MULTIINSTANCE\_MANAGEMENT

\item {} 
\sphinxAtStartPar
ENERGY\_SYSTEM\_SHARING

\item {} 
\sphinxAtStartPar
CLIMATIC\_CHAMBER\_SHARING

\item {} 
\sphinxAtStartPar
COOLANT\_COND\_SHARING

\end{itemize}

\item {} 
\sphinxAtStartPar
SCALE Devices
\begin{itemize}
\item {} 
\sphinxAtStartPar
ENERGY\_SYSTEM\_X.\sphinxstyleemphasis{\sphinxstylestrong{your device}}

\item {} 
\sphinxAtStartPar
CLIMATIC\_CHAMBER\_X.\sphinxstyleemphasis{\sphinxstylestrong{your device}}

\item {} 
\sphinxAtStartPar
COOLANT\_COND\_X.\sphinxstyleemphasis{\sphinxstylestrong{your device}}

\end{itemize}

\item {} 
\sphinxAtStartPar
Multi instance
\begin{itemize}
\item {} 
\sphinxAtStartPar
Multi Instance code for 4, 8, 16 or 32 instances

\end{itemize}

\end{itemize}

\sphinxAtStartPar
If codes are missing please contact \sphinxhref{mailto:hotline@fev.com}{hotline@fev.com}


\chapter{Base configuration}
\label{\detokenize{04_Base-configuration:base-configuration}}\label{\detokenize{04_Base-configuration::doc}}
\begin{sphinxadmonition}{important}{Important:}
\sphinxAtStartPar
This documentation will not develop the hardware and feature configuration.
This has to be done previously.

\sphinxAtStartPar
Nevertheless, it will be possible to test all equipement one by one when it will be possible to start MORPHEE® master instance.
\end{sphinxadmonition}


\section{Principle}
\label{\detokenize{04_Base-configuration:principle}}
\sphinxAtStartPar
Concept of the SCALE Battery configuration is to use most of the equipement in the master instance and to give the possibility for each instance to attach to one (or more) equipement through ‘Sharing’ components;

\begin{sphinxuseclass}{full-width}
\noindent\sphinxincludegraphics{{Battery_Concept}.png}

\end{sphinxuseclass}
\sphinxAtStartPar
The Sharing equipements take care of all present equipement of the familly;

\sphinxAtStartPar
ENERGY\_SYSTEM\_SHARING
\begin{itemize}
\item {} 
\sphinxAtStartPar
ENERGY\_SYSTEM\_1

\item {} 
\sphinxAtStartPar
ENERGY\_SYSTEM\_2

\item {} 
\sphinxAtStartPar
…

\item {} 
\sphinxAtStartPar
ENERGY\_SYSTEM\_x

\end{itemize}


\section{Component role}
\label{\detokenize{04_Base-configuration:component-role}}

\subsection{Definition}
\label{\detokenize{04_Base-configuration:definition}}
\sphinxAtStartPar
Component role are defined below:

\begin{sphinxuseclass}{full-width}

\begin{savenotes}\sphinxattablestart
\centering
\begin{tabulary}{\linewidth}[t]{|T|T|T|T|T|}
\hline
\sphinxstyletheadfamily 
\sphinxAtStartPar
Component
&
\sphinxAtStartPar

&\sphinxstyletheadfamily 
\sphinxAtStartPar
Master
&\sphinxstyletheadfamily 
\sphinxAtStartPar
Inst.
&\sphinxstyletheadfamily 
\sphinxAtStartPar
Description
\\
\hline
\sphinxAtStartPar
\sphinxstylestrong{MANAGEMENT}
&
\sphinxAtStartPar
\sphinxstyleemphasis{SCALE}
&
\sphinxAtStartPar
\sphinxstylestrong{X}
&
\sphinxAtStartPar
\sphinxstylestrong{X}
&
\sphinxAtStartPar
Global management of all SCALE and non SCALE component in the configuration. Must be in all configuration
\\
\hline
\sphinxAtStartPar
\sphinxstylestrong{CONTROL\_MANAGEMENT.BATTERY}
&
\sphinxAtStartPar
\sphinxstyleemphasis{SCALE}
&
\sphinxAtStartPar

&
\sphinxAtStartPar
\sphinxstylestrong{X}
&
\sphinxAtStartPar
Management of the different control mode for battery;
\\
\hline
\sphinxAtStartPar
\sphinxstylestrong{MEASUREMENT\_MANAGEMENT.BATTERY}
&
\sphinxAtStartPar
\sphinxstyleemphasis{SCALE}
&
\sphinxAtStartPar

&
\sphinxAtStartPar
\sphinxstylestrong{X}
&
\sphinxAtStartPar
Management of acquisition for battery;
\\
\hline
\sphinxAtStartPar
\sphinxstylestrong{MULTIINSTANCE\_MANAGEMENT}
&
\sphinxAtStartPar
\sphinxstyleemphasis{SCALE}
&
\sphinxAtStartPar
\sphinxstylestrong{X}
&
\sphinxAtStartPar

&
\sphinxAtStartPar
Handling of the configuration (see below) and instances
\\
\hline
\sphinxAtStartPar
\sphinxstylestrong{ACQUISITION}
&
\sphinxAtStartPar

&
\sphinxAtStartPar
\sphinxstylestrong{X}
&
\sphinxAtStartPar
(\sphinxstylestrong{X})
&
\sphinxAtStartPar
Component for “normed name” customer channels. This is the main place to define: Link to hardware channel Securities Calculation Acquisition plan
\\
\hline
\sphinxAtStartPar
\sphinxstylestrong{HARDWARE}
&
\sphinxAtStartPar

&
\sphinxAtStartPar
\sphinxstylestrong{X}
&
\sphinxAtStartPar

&
\sphinxAtStartPar
Component to place hardware channels of the configuration. Act as a simple passive component to concentrate acquisition channels.
\\
\hline
\sphinxAtStartPar
\sphinxstyleemphasis{\sphinxstylestrong{Customer specific component}}
&
\sphinxAtStartPar

&
\sphinxAtStartPar
\sphinxstylestrong{X}
&
\sphinxAtStartPar
(\sphinxstylestrong{X})
&
\sphinxAtStartPar
All needed component for customer configuration, such as; PLC Barcode Various interface …
\\
\hline
\sphinxAtStartPar
\sphinxstylestrong{BMS}
&
\sphinxAtStartPar
\sphinxstyleemphasis{SCALE}
&
\sphinxAtStartPar

&
\sphinxAtStartPar
\sphinxstylestrong{X}
&
\sphinxAtStartPar
Communication with the battery BMS, adressed either in CAN or FDX protocol. Frame given by SCALE architecture, but possibility to ADD new methods and channels if there is a need to add customer specific command (father method)
\\
\hline
\sphinxAtStartPar
\sphinxstylestrong{DUT}
&
\sphinxAtStartPar
\sphinxstyleemphasis{SCALE}
&
\sphinxAtStartPar

&
\sphinxAtStartPar
\sphinxstylestrong{X}
&
\sphinxAtStartPar
DUT description. Frame given by SCALE architecture, but possibility to ADD new parameters if needed.
\\
\hline
\sphinxAtStartPar
\sphinxstylestrong{ENERGY\_SYSTEM\_SHARING}
&
\sphinxAtStartPar
\sphinxstyleemphasis{SCALE}
&
\sphinxAtStartPar
.MASTER
&
\sphinxAtStartPar
.SLAVE
&
\sphinxAtStartPar
Management of different ENERGY\_SYSTEM; In link with the instance In regards to configuration
\\
\hline
\sphinxAtStartPar
\sphinxstylestrong{ENERGY\_SYSTEM\_X}
&
\sphinxAtStartPar
\sphinxstyleemphasis{SCALE}
&
\sphinxAtStartPar
\sphinxstylestrong{X}
&
\sphinxAtStartPar

&
\sphinxAtStartPar
SCALE energy system devices
\\
\hline
\sphinxAtStartPar
\sphinxstylestrong{CLIMATIC\_CHAMBER\_SHARING}
&
\sphinxAtStartPar
\sphinxstyleemphasis{SCALE}
&
\sphinxAtStartPar
.MASTER
&
\sphinxAtStartPar
.SLAVE
&
\sphinxAtStartPar
Management of different CLIMATIC\_CHAMBER; In link with the instance In regards to configuration
\\
\hline
\sphinxAtStartPar
\sphinxstylestrong{CLIMATIC\_CHAMBER\_X}
&
\sphinxAtStartPar
\sphinxstyleemphasis{SCALE}
&
\sphinxAtStartPar
\sphinxstylestrong{X}
&
\sphinxAtStartPar

&
\sphinxAtStartPar
SCALE climatic chamber devices
\\
\hline
\sphinxAtStartPar
\sphinxstylestrong{COOLAND\_COND\_SHARING}
&
\sphinxAtStartPar
\sphinxstyleemphasis{SCALE}
&
\sphinxAtStartPar
.MASTER
&
\sphinxAtStartPar
.SLAVE
&
\sphinxAtStartPar
Management of different COOLAND\_COND; In link with the instance In regards to configuration
\\
\hline
\sphinxAtStartPar
\sphinxstylestrong{COOLAND\_COND\_X}
&
\sphinxAtStartPar
\sphinxstyleemphasis{SCALE}
&
\sphinxAtStartPar
\sphinxstylestrong{X}
&
\sphinxAtStartPar

&
\sphinxAtStartPar
SCALE coolant conditionning devices
\\
\hline
\sphinxAtStartPar
\sphinxstyleemphasis{\sphinxstylestrong{SCADA\_CLIENT\_MASTER}}
&
\sphinxAtStartPar
\sphinxstyleemphasis{SCALE}
&
\sphinxAtStartPar
\sphinxstylestrong{X}
&
\sphinxAtStartPar

&
\sphinxAtStartPar
Can be used only with SCADA
\\
\hline
\sphinxAtStartPar
\sphinxstyleemphasis{\sphinxstylestrong{SCADA\_CLIENT\_INSTANCE}}
&
\sphinxAtStartPar
\sphinxstyleemphasis{SCALE}
&
\sphinxAtStartPar

&
\sphinxAtStartPar
\sphinxstylestrong{X}
&
\sphinxAtStartPar
Can be used only with SCADA
\\
\hline
\sphinxAtStartPar

&
\sphinxAtStartPar

&
\sphinxAtStartPar

&
\sphinxAtStartPar

&
\sphinxAtStartPar

\\
\hline
\end{tabulary}
\par
\sphinxattableend\end{savenotes}

\end{sphinxuseclass}

\subsection{Configuration in UEditor}
\label{\detokenize{04_Base-configuration:configuration-in-ueditor}}
\sphinxAtStartPar
Here is an example of a standard Bench configuration;

\sphinxAtStartPar
\sphinxincludegraphics{{UEditor_Bench_config}.png}

\begin{sphinxadmonition}{hint}{Hint:}
\sphinxAtStartPar
To avoid re\sphinxhyphen{}design of the Dashborad, it makes sense to always use all equipement in the Bench configuration and make the link ‘Optional’ if they are not used;

\sphinxAtStartPar
\sphinxincludegraphics{{UEditor_Link_Optionnal}.png}
\end{sphinxadmonition}


\subsection{MultiInstance configuration}
\label{\detokenize{04_Base-configuration:multiinstance-configuration}}
\sphinxAtStartPar
Before starting, you need also to configure the MULTIINSTANCE\_MANAGEMENT component:

\sphinxAtStartPar
\sphinxincludegraphics{{UEditor_MM_Configuration}.png}

\sphinxAtStartPar
By default, you can configure as follow;


\begin{savenotes}\sphinxattablestart
\centering
\begin{tabulary}{\linewidth}[t]{|T|T|T|}
\hline
\sphinxstyletheadfamily 
\sphinxAtStartPar
Parameter
&\sphinxstyletheadfamily 
\sphinxAtStartPar
Description
&\sphinxstyletheadfamily 
\sphinxAtStartPar
Value
\\
\hline
\sphinxAtStartPar
P\_MIM.NB\_INSTANCES
&
\sphinxAtStartPar
Quantity of MORPHEE Instance
&
\sphinxAtStartPar
8
\\
\hline
\sphinxAtStartPar
P\_MIM.DATA\_HANDLING
&
\sphinxAtStartPar
Management of parameters
&
\sphinxAtStartPar
By MORPHEE
\\
\hline
\end{tabulary}
\par
\sphinxattableend\end{savenotes}


\subsection{Configuration without any hardware}
\label{\detokenize{04_Base-configuration:configuration-without-any-hardware}}
\sphinxAtStartPar
Example of Configuration without any hardware and simulated components:


\begin{savenotes}\sphinxattablestart
\centering
\begin{tabulary}{\linewidth}[t]{|T|}
\hline
\sphinxstyletheadfamily 
\sphinxAtStartPar
Master configuration (In Bench cfg)
\\
\hline
\sphinxAtStartPar
MANAGEMENT
\\
\hline
\sphinxAtStartPar
MULTIINSTANCE\_MANAGEMENT
\\
\hline
\sphinxAtStartPar
ENERGY\_SYSTEM\_SHARING.\sphinxstyleemphasis{\sphinxstylestrong{MASTER}}
\\
\hline
\sphinxAtStartPar
ENERGY\_SYSTEM\_1.SIMU\_BASIC
\\
\hline
\sphinxAtStartPar
ENERGY\_SYSTEM\_2.SIMU\_BASIC
\\
\hline
\sphinxAtStartPar
ENERGY\_SYSTEM\_3.SIMU\_BASIC
\\
\hline
\sphinxAtStartPar
ENERGY\_SYSTEM\_4.SIMU\_BASIC
\\
\hline
\sphinxAtStartPar
CLIMATIC\_CHAMBER\_SHARING.\sphinxstyleemphasis{\sphinxstylestrong{MASTER}}
\\
\hline
\sphinxAtStartPar
CLIMATIC\_CHAMBER\_1.SIMULATION
\\
\hline
\sphinxAtStartPar
CLIMATIC\_CHAMBER\_2.SIMULATION
\\
\hline
\sphinxAtStartPar
COOLANT\_COND\_SHARING.\sphinxstyleemphasis{\sphinxstylestrong{MASTER}}
\\
\hline
\sphinxAtStartPar
COOLANT\_COND\_1.SIMULATION
\\
\hline
\sphinxAtStartPar

\\
\hline
\end{tabulary}
\par
\sphinxattableend\end{savenotes}

\begin{sphinxuseclass}{full-width}

\begin{savenotes}\sphinxattablestart
\centering
\begin{tabulary}{\linewidth}[t]{|T|T|T|}
\hline
\sphinxstyletheadfamily 
\sphinxAtStartPar
Master configuration (In Bench cfg)
&\sphinxstyletheadfamily 
\sphinxAtStartPar
Instance 1 configuration (In Test cfg)
&\sphinxstyletheadfamily 
\sphinxAtStartPar
Instance n configuration (In Test cfg)
\\
\hline
\sphinxAtStartPar
MANAGEMENT
&
\sphinxAtStartPar
MANAGEMENT
&
\sphinxAtStartPar
MANAGEMENT
\\
\hline
\sphinxAtStartPar
MULTIINSTANCE\_MANAGEMENT
&
\sphinxAtStartPar
CONTROL\_MANAGEMENT.BATTERY
&
\sphinxAtStartPar
CONTROL\_MANAGEMENT.BATTERY
\\
\hline
\sphinxAtStartPar
ENERGY\_SYSTEM\_SHARING.\sphinxstyleemphasis{\sphinxstylestrong{MASTER}}
&
\sphinxAtStartPar
ENERGY\_SYSTEM\_SHARING.\sphinxstyleemphasis{\sphinxstylestrong{SLAVE}}
&
\sphinxAtStartPar
ENERGY\_SYSTEM\_SHARING.\sphinxstyleemphasis{\sphinxstylestrong{SLAVE}}
\\
\hline
\sphinxAtStartPar
ENERGY\_SYSTEM\_1.SIMU\_BASIC
&
\sphinxAtStartPar

&
\sphinxAtStartPar

\\
\hline
\sphinxAtStartPar
ENERGY\_SYSTEM\_2.SIMU\_BASIC
&
\sphinxAtStartPar

&
\sphinxAtStartPar

\\
\hline
\sphinxAtStartPar
ENERGY\_SYSTEM\_3.SIMU\_BASIC
&
\sphinxAtStartPar

&
\sphinxAtStartPar

\\
\hline
\sphinxAtStartPar
ENERGY\_SYSTEM\_4.SIMU\_BASIC
&
\sphinxAtStartPar

&
\sphinxAtStartPar

\\
\hline
\sphinxAtStartPar
CLIMATIC\_CHAMBER\_SHARING.\sphinxstyleemphasis{\sphinxstylestrong{MASTER}}
&
\sphinxAtStartPar
CLIMATIC\_CHAMBER\_SHARING.\sphinxstyleemphasis{\sphinxstylestrong{SLAVE}}
&
\sphinxAtStartPar
CLIMATIC\_CHAMBER\_SHARING.\sphinxstyleemphasis{\sphinxstylestrong{MASTER}}
\\
\hline
\sphinxAtStartPar
CLIMATIC\_CHAMBER\_1.SIMULATION
&
\sphinxAtStartPar

&
\sphinxAtStartPar

\\
\hline
\sphinxAtStartPar
CLIMATIC\_CHAMBER\_2.SIMULATION
&
\sphinxAtStartPar

&
\sphinxAtStartPar

\\
\hline
\sphinxAtStartPar
COOLANT\_COND\_SHARING.\sphinxstyleemphasis{\sphinxstylestrong{MASTER}}
&
\sphinxAtStartPar
COOLANT\_COND\_SHARING.\sphinxstyleemphasis{\sphinxstylestrong{SLAVE}}
&
\sphinxAtStartPar
COOLANT\_COND\_SHARING.\sphinxstyleemphasis{\sphinxstylestrong{SLAVE}}
\\
\hline
\sphinxAtStartPar
COOLANT\_COND\_1.SIMULATION
&
\sphinxAtStartPar

&
\sphinxAtStartPar

\\
\hline
\sphinxAtStartPar

&
\sphinxAtStartPar

&
\sphinxAtStartPar

\\
\hline
\end{tabulary}
\par
\sphinxattableend\end{savenotes}

\end{sphinxuseclass}
\begin{sphinxadmonition}{note}{Note:}
\sphinxAtStartPar
All component existing in your configuration but not present in this table can be put as ‘Optional’
\end{sphinxadmonition}


\subsection{Configuration with hardware}
\label{\detokenize{04_Base-configuration:configuration-with-hardware}}
\sphinxAtStartPar
The configuration with hardware is not so different. You just have to replace ‘Simulated’ equipments with real one, and add your own component if needed (Acquisition, Hardware, etc…).

\sphinxAtStartPar
The table below is not exhaustive but show the configuration concept by using hardware:


\begin{savenotes}\sphinxattablestart
\centering
\begin{tabulary}{\linewidth}[t]{|T|}
\hline
\sphinxstyletheadfamily 
\sphinxAtStartPar
Master configuration (In Bench cfg)
\\
\hline
\sphinxAtStartPar
MANAGEMENT
\\
\hline
\sphinxAtStartPar
MULTIINSTANCE\_MANAGEMENT
\\
\hline
\sphinxAtStartPar
ENERGY\_SYSTEM\_SHARING.\sphinxstyleemphasis{\sphinxstylestrong{MASTER}}
\\
\hline
\sphinxAtStartPar
ENERGY\_SYSTEM\_1.\sphinxstyleemphasis{Real device}
\\
\hline
\sphinxAtStartPar
ENERGY\_SYSTEM\_2.\sphinxstyleemphasis{Real device}
\\
\hline
\sphinxAtStartPar
ENERGY\_SYSTEM\_3.\sphinxstyleemphasis{Real device}
\\
\hline
\sphinxAtStartPar
ENERGY\_SYSTEM\_4.\sphinxstyleemphasis{Real device}
\\
\hline
\sphinxAtStartPar
CLIMATIC\_CHAMBER\_SHARING.\sphinxstyleemphasis{\sphinxstylestrong{MASTER}}
\\
\hline
\sphinxAtStartPar
CLIMATIC\_CHAMBER\_1.\sphinxstyleemphasis{Real device}
\\
\hline
\sphinxAtStartPar
CLIMATIC\_CHAMBER\_2.\sphinxstyleemphasis{Real device}
\\
\hline
\sphinxAtStartPar
COOLANT\_COND\_SHARING.\sphinxstyleemphasis{\sphinxstylestrong{MASTER}}
\\
\hline
\sphinxAtStartPar
COOLANT\_COND\_1.\sphinxstyleemphasis{Real device}
\\
\hline
\sphinxAtStartPar

\\
\hline
\end{tabulary}
\par
\sphinxattableend\end{savenotes}


\section{First start}
\label{\detokenize{04_Base-configuration:first-start}}
\begin{sphinxadmonition}{hint}{Hint:}
\sphinxAtStartPar
We recommend, for the first start, to use the simulated components and to limit the number of instances.
\end{sphinxadmonition}

\sphinxAtStartPar
When you have reach this point, you are ready to start SCALE Battery master instance.

\sphinxAtStartPar
You should see:
\sphinxincludegraphics{{MO_Main_Screen}.png}


\chapter{Runtime configuration}
\label{\detokenize{05_SCALE_Battery-configuration:runtime-configuration}}\label{\detokenize{05_SCALE_Battery-configuration::doc}}
\begin{sphinxadmonition}{important}{Important:}
\sphinxAtStartPar
If you read this section, this means you succeed to start MORPHEE main instance. If not, please go back to previous section.
\end{sphinxadmonition}


\section{Dashboard}
\label{\detokenize{05_SCALE_Battery-configuration:dashboard}}
\sphinxAtStartPar
Main Dashboard can be modified in the Bench configuration. By default, it displays the different slots, the climatic chamber information, The climatic chamber sharing configuration and the Energy system sharing configuration.

\sphinxAtStartPar
\sphinxincludegraphics{{MO_Main_Screen_commented}.png}


\subsection{Slots configuration}
\label{\detokenize{05_SCALE_Battery-configuration:slots-configuration}}
\sphinxAtStartPar
You can modify the configuration by pushing the ‘Edit configuration’ button.

\sphinxAtStartPar
\sphinxincludegraphics{{MO_Cfg_A}.png}

\sphinxAtStartPar
It will display all the Slot and you will be able to modify each slot configuration.


\subsection{Channel affectation}
\label{\detokenize{05_SCALE_Battery-configuration:channel-affectation}}
\sphinxAtStartPar
You can select the channels you want to affect to the Slot.

\sphinxAtStartPar
\sphinxincludegraphics{{MO_Cfg_B}.png}

\sphinxAtStartPar
By default, each slot has its own channel;


\begin{savenotes}\sphinxattablestart
\centering
\begin{tabulary}{\linewidth}[t]{|T|T|}
\hline
\sphinxstyletheadfamily 
\sphinxAtStartPar
Slot
&\sphinxstyletheadfamily 
\sphinxAtStartPar
Electric Channel
\\
\hline
\sphinxAtStartPar
Slot 1
&
\sphinxAtStartPar
Channel 1
\\
\hline
\sphinxAtStartPar
Slot 2
&
\sphinxAtStartPar
Channel 2
\\
\hline
\sphinxAtStartPar
Slot 3
&
\sphinxAtStartPar
Channel 3
\\
\hline
\sphinxAtStartPar
Slot 4
&
\sphinxAtStartPar
Channel 4
\\
\hline
\sphinxAtStartPar
Slot 5
&
\sphinxAtStartPar
Channel 5
\\
\hline
\sphinxAtStartPar
…
&
\sphinxAtStartPar
…
\\
\hline
\sphinxAtStartPar
Slot X
&
\sphinxAtStartPar
Channel X
\\
\hline
\sphinxAtStartPar

&
\sphinxAtStartPar

\\
\hline
\end{tabulary}
\par
\sphinxattableend\end{savenotes}

\sphinxAtStartPar
But if ENERGY\_SYSTEM Component allows it, we can use several channels for one slot;


\begin{savenotes}\sphinxattablestart
\centering
\begin{tabulary}{\linewidth}[t]{|T|T|}
\hline
\sphinxstyletheadfamily 
\sphinxAtStartPar
Slot
&\sphinxstyletheadfamily 
\sphinxAtStartPar
Electric Channel
\\
\hline
\sphinxAtStartPar
Slot 1
&
\sphinxAtStartPar
Channel 1 + Channel 2
\\
\hline
\sphinxAtStartPar
Slot 2
&
\sphinxAtStartPar
\sphinxstyleemphasis{No more available}
\\
\hline
\sphinxAtStartPar
Slot 3
&
\sphinxAtStartPar
Channel 3 + Channel 4 + Channel 5
\\
\hline
\sphinxAtStartPar
Slot 4
&
\sphinxAtStartPar
\sphinxstyleemphasis{No more available}
\\
\hline
\sphinxAtStartPar
Slot 5
&
\sphinxAtStartPar
\sphinxstyleemphasis{No more available}
\\
\hline
\sphinxAtStartPar
…
&
\sphinxAtStartPar
…
\\
\hline
\sphinxAtStartPar
Slot X
&
\sphinxAtStartPar
Channel X
\\
\hline
\sphinxAtStartPar

&
\sphinxAtStartPar

\\
\hline
\end{tabulary}
\par
\sphinxattableend\end{savenotes}

\sphinxAtStartPar
etc…

\sphinxAtStartPar
You can modify the configuration by clicking in on the right and left button:


\begin{savenotes}\sphinxattablestart
\centering
\begin{tabulary}{\linewidth}[t]{|T|T|T|}
\hline
\sphinxstyletheadfamily 
\sphinxAtStartPar
No channel
&\sphinxstyletheadfamily 
\sphinxAtStartPar
One channel
&\sphinxstyletheadfamily 
\sphinxAtStartPar
Two channels (if possible)
\\
\hline
\sphinxAtStartPar
\sphinxincludegraphics{{MO_Cfg_Channel_none}.png}
&
\sphinxAtStartPar
\sphinxincludegraphics{{MO_Cfg_Channel_1}.png}
&
\sphinxAtStartPar
\sphinxincludegraphics{{MO_Cfg_Channel_2}.png}
\\
\hline
\sphinxAtStartPar

&
\sphinxAtStartPar

&
\sphinxAtStartPar

\\
\hline
\end{tabulary}
\par
\sphinxattableend\end{savenotes}

\begin{sphinxadmonition}{caution}{Caution:}
\sphinxAtStartPar
If a channel is used by a slot, it is not anymore available for the next slot.
\end{sphinxadmonition}


\subsection{Climatic chamber selection}
\label{\detokenize{05_SCALE_Battery-configuration:climatic-chamber-selection}}
\sphinxAtStartPar
When you have affected a channel, you can precise if the slot is in a climatic chamber, and if yes, in which one;

\sphinxAtStartPar
\sphinxincludegraphics{{MO_Cfg_D}.png}

\sphinxAtStartPar
The list of Climatic chamber depends on the configuration:

\sphinxAtStartPar
\sphinxincludegraphics{{MO_Cfg_Clim}.png}


\subsection{Cooling system selection}
\label{\detokenize{05_SCALE_Battery-configuration:cooling-system-selection}}
\sphinxAtStartPar
If battery is using a cooling system, it can also be defined here;

\sphinxAtStartPar
\sphinxincludegraphics{{MO_Cfg_C}.png}

\sphinxAtStartPar
The list of Coolant conditionning circuit appears in the list:

\sphinxAtStartPar
\sphinxincludegraphics{{MO_Cfg_Cooling}.png}


\section{Climatic chamber sharing}
\label{\detokenize{05_SCALE_Battery-configuration:climatic-chamber-sharing}}
\sphinxAtStartPar
As it is possible to have several slots inside the same climatic chamber, it is necessary to describe how the system will react if different setpoint arrives in the master;

\sphinxAtStartPar
\sphinxincludegraphics{{MO_Cfg_Clim_sharing}.png}

\sphinxAtStartPar
The possibilities are:


\begin{savenotes}\sphinxattablestart
\centering
\begin{tabulary}{\linewidth}[t]{|T|T|}
\hline

\sphinxAtStartPar

&
\sphinxAtStartPar

\\
\hline
\sphinxAtStartPar
\sphinxstylestrong{Managed by master only}
&
\sphinxAtStartPar
The temperature setpoint is managed by the master. Every test from instances, even if sending a setpoint, will not be taken into account
\\
\hline
\sphinxAtStartPar
\sphinxstylestrong{Last T°C value received}
&
\sphinxAtStartPar
The last temperature received will be applied, wherever and whenever it arrives.
\\
\hline
\sphinxAtStartPar
\sphinxstylestrong{Synchronized on T°C}
&
\sphinxAtStartPar
Wait that all runing instances send the same setpoint to apply it.  All the test on the different instances will be on hold as long as setpoint is not applyed.
\\
\hline
\sphinxAtStartPar
\sphinxstylestrong{Handled by instance \sphinxstyleemphasis{X}}
&
\sphinxAtStartPar
Only setpoint coming from Instance \sphinxstyleemphasis{\sphinxstylestrong{X}} will be applied.
\\
\hline
\sphinxAtStartPar

&
\sphinxAtStartPar

\\
\hline
\end{tabulary}
\par
\sphinxattableend\end{savenotes}


\section{Energy System sharing}
\label{\detokenize{05_SCALE_Battery-configuration:energy-system-sharing}}
\sphinxAtStartPar
The Energy System Sharing window will display actual configuration with available slots:

\sphinxAtStartPar
\sphinxincludegraphics{{MO_Cfg_Energy_sharing}.png}

\sphinxAtStartPar
For every Energy System used in the {\hyperref[\detokenize{04_Base-configuration::doc}]{\sphinxcrossref{\DUrole{doc,std,std-doc}{configuration}}}}, it will display:
\begin{itemize}
\item {} 
\sphinxAtStartPar
Name of the component

\item {} 
\sphinxAtStartPar
Number of channels in this component

\end{itemize}


\part{Utilization}


\chapter{Starting an Instance}
\label{\detokenize{06_Starting_An_instance:starting-an-instance}}\label{\detokenize{06_Starting_An_instance::doc}}

\section{The first start}
\label{\detokenize{06_Starting_An_instance:the-first-start}}
\sphinxAtStartPar
When your configuration is complete, you are ready to start your first instance. By default, you should see some battery description and at least one basic test (EMPTY\_TEST\_NEWCMA):

\sphinxAtStartPar
\sphinxincludegraphics{{MO_Inst_First_start}.png}
\begin{enumerate}
\sphinxsetlistlabels{\arabic}{enumi}{enumii}{}{.}%
\item {} 
\sphinxAtStartPar
Select a battery definition

\item {} 
\sphinxAtStartPar
Select the test: EMPTY\_TEST

\item {} 
\sphinxAtStartPar
Push the ‘PLAY BUTTON’

\end{enumerate}

\sphinxAtStartPar
This should result in:
\sphinxincludegraphics{{MO_Inst_Main_screen}.png}
\begin{itemize}
\item {} 
\sphinxAtStartPar
In the top left, you should see the instance number

\item {} 
\sphinxAtStartPar
On main screen:
\begin{itemize}
\item {} 
\sphinxAtStartPar
The control Management interface

\item {} 
\sphinxAtStartPar
The coolant conditioning interface

\item {} 
\sphinxAtStartPar
The climatic chamber interface

\end{itemize}

\end{itemize}


\section{Testing the functionnalities}
\label{\detokenize{06_Starting_An_instance:testing-the-functionnalities}}

\subsection{The Control Management}
\label{\detokenize{06_Starting_An_instance:the-control-management}}
\sphinxAtStartPar
Once the instance started, you should \sphinxstylestrong{PREPARE} the different component. For this, push the \sphinxstylestrong{PREPARE} button on the top right.

\sphinxAtStartPar
When prepared, you are ready to send commands to the related \sphinxstylestrong{ENERGY\_SYSTEM}

\begin{sphinxadmonition}{note}{Note:}
\sphinxAtStartPar
\sphinxstylestrong{Control Management}

\begin{sphinxuseclass}{sphinx-bs}
\begin{sphinxuseclass}{container}
\begin{sphinxuseclass}{pb-4}
\begin{sphinxuseclass}{row}
\begin{sphinxuseclass}{d-flex}
\begin{sphinxuseclass}{col-lg-6}
\begin{sphinxuseclass}{col-md-6}
\begin{sphinxuseclass}{col-sm-6}
\begin{sphinxuseclass}{col-xs-12}
\begin{sphinxuseclass}{p-2}
\begin{sphinxuseclass}{card}
\begin{sphinxuseclass}{w-100}
\begin{sphinxuseclass}{shadow}
\begin{sphinxuseclass}{card-body}
\sphinxAtStartPar
\sphinxincludegraphics{{MO_Inst_CMA}.png}

\end{sphinxuseclass}
\end{sphinxuseclass}
\end{sphinxuseclass}
\end{sphinxuseclass}
\end{sphinxuseclass}
\end{sphinxuseclass}
\end{sphinxuseclass}
\end{sphinxuseclass}
\end{sphinxuseclass}
\end{sphinxuseclass}
\begin{sphinxuseclass}{d-flex}
\begin{sphinxuseclass}{col-lg-6}
\begin{sphinxuseclass}{col-md-6}
\begin{sphinxuseclass}{col-sm-6}
\begin{sphinxuseclass}{col-xs-12}
\begin{sphinxuseclass}{p-2}
\begin{sphinxuseclass}{card}
\begin{sphinxuseclass}{w-100}
\begin{sphinxuseclass}{shadow}
\begin{sphinxuseclass}{card-body}\begin{itemize}
\item {} 
\sphinxAtStartPar
States:
\begin{itemize}
\item {} 
\sphinxAtStartPar
\sphinxstylestrong{Rest}: Battery is in Idle mode

\item {} 
\sphinxAtStartPar
\sphinxstylestrong{Current}: Battery is in Current mode

\item {} 
\sphinxAtStartPar
\sphinxstylestrong{Volatge}: Battery is in Voltage mode

\item {} 
\sphinxAtStartPar
\sphinxstylestrong{Power}: Battery is in Power mode

\item {} 
\sphinxAtStartPar
\sphinxstyleemphasis{Each button will select the related mode}

\end{itemize}

\item {} 
\sphinxAtStartPar
When the mode is set:
\begin{itemize}
\item {} 
\sphinxAtStartPar
Setpoint can be send through S\_CMA.\sphinxstylestrong{f}\_SET (where \sphinxstylestrong{f} is either I, V or P)

\item {} 
\sphinxAtStartPar
Measurement are read in R\_CMA.\sphinxstylestrong{f}\_ACT (where \sphinxstylestrong{f} is either I, V or P)

\end{itemize}

\item {} 
\sphinxAtStartPar
Integrated function:
\begin{itemize}
\item {} 
\sphinxAtStartPar
SET\_CC: Integrated command to send current setpoint with voltage and duration limit.

\item {} 
\sphinxAtStartPar
SET\_CV: Integrated command to send voltage setpoint with current and duration limit.

\item {} 
\sphinxAtStartPar
SET\_CP: Integrated command to send power setpoint with voltage and duration limit.

\end{itemize}

\end{itemize}

\end{sphinxuseclass}
\end{sphinxuseclass}
\end{sphinxuseclass}
\end{sphinxuseclass}
\end{sphinxuseclass}
\end{sphinxuseclass}
\end{sphinxuseclass}
\end{sphinxuseclass}
\end{sphinxuseclass}
\end{sphinxuseclass}
\end{sphinxuseclass}
\end{sphinxuseclass}
\end{sphinxuseclass}
\end{sphinxuseclass}\end{sphinxadmonition}


\subsection{The coolant confitioning}
\label{\detokenize{06_Starting_An_instance:the-coolant-confitioning}}
\begin{sphinxadmonition}{note}{Note:}
\sphinxAtStartPar
\sphinxstylestrong{Coolant conditionning}

\begin{sphinxuseclass}{sphinx-bs}
\begin{sphinxuseclass}{container}
\begin{sphinxuseclass}{pb-4}
\begin{sphinxuseclass}{row}
\begin{sphinxuseclass}{d-flex}
\begin{sphinxuseclass}{col-lg-6}
\begin{sphinxuseclass}{col-md-6}
\begin{sphinxuseclass}{col-sm-6}
\begin{sphinxuseclass}{col-xs-12}
\begin{sphinxuseclass}{p-2}
\begin{sphinxuseclass}{card}
\begin{sphinxuseclass}{w-100}
\begin{sphinxuseclass}{shadow}
\begin{sphinxuseclass}{card-body}
\sphinxAtStartPar
\sphinxincludegraphics{{MO_Inst_CCS}.png}

\end{sphinxuseclass}
\end{sphinxuseclass}
\end{sphinxuseclass}
\end{sphinxuseclass}
\end{sphinxuseclass}
\end{sphinxuseclass}
\end{sphinxuseclass}
\end{sphinxuseclass}
\end{sphinxuseclass}
\end{sphinxuseclass}
\begin{sphinxuseclass}{d-flex}
\begin{sphinxuseclass}{col-lg-6}
\begin{sphinxuseclass}{col-md-6}
\begin{sphinxuseclass}{col-sm-6}
\begin{sphinxuseclass}{col-xs-12}
\begin{sphinxuseclass}{p-2}
\begin{sphinxuseclass}{card}
\begin{sphinxuseclass}{w-100}
\begin{sphinxuseclass}{shadow}
\begin{sphinxuseclass}{card-body}\begin{itemize}
\item {} 
\sphinxAtStartPar
Start the conditioning
\begin{itemize}
\item {} 
\sphinxAtStartPar
When started, different setpoint are take into account

\end{itemize}

\item {} 
\sphinxAtStartPar
Stop the conditioning

\end{itemize}

\end{sphinxuseclass}
\end{sphinxuseclass}
\end{sphinxuseclass}
\end{sphinxuseclass}
\end{sphinxuseclass}
\end{sphinxuseclass}
\end{sphinxuseclass}
\end{sphinxuseclass}
\end{sphinxuseclass}
\end{sphinxuseclass}
\end{sphinxuseclass}
\end{sphinxuseclass}
\end{sphinxuseclass}
\end{sphinxuseclass}\end{sphinxadmonition}


\subsection{The climatic chamber}
\label{\detokenize{06_Starting_An_instance:the-climatic-chamber}}
\begin{sphinxadmonition}{note}{Note:}
\sphinxAtStartPar
\sphinxstylestrong{Climatic Chamber}

\begin{sphinxuseclass}{sphinx-bs}
\begin{sphinxuseclass}{container}
\begin{sphinxuseclass}{pb-4}
\begin{sphinxuseclass}{row}
\begin{sphinxuseclass}{d-flex}
\begin{sphinxuseclass}{col-lg-6}
\begin{sphinxuseclass}{col-md-6}
\begin{sphinxuseclass}{col-sm-6}
\begin{sphinxuseclass}{col-xs-12}
\begin{sphinxuseclass}{p-2}
\begin{sphinxuseclass}{card}
\begin{sphinxuseclass}{w-100}
\begin{sphinxuseclass}{shadow}
\begin{sphinxuseclass}{card-body}
\sphinxAtStartPar
\sphinxincludegraphics{{MO_Inst_CCHAS}.png}

\end{sphinxuseclass}
\end{sphinxuseclass}
\end{sphinxuseclass}
\end{sphinxuseclass}
\end{sphinxuseclass}
\end{sphinxuseclass}
\end{sphinxuseclass}
\end{sphinxuseclass}
\end{sphinxuseclass}
\end{sphinxuseclass}
\begin{sphinxuseclass}{d-flex}
\begin{sphinxuseclass}{col-lg-6}
\begin{sphinxuseclass}{col-md-6}
\begin{sphinxuseclass}{col-sm-6}
\begin{sphinxuseclass}{col-xs-12}
\begin{sphinxuseclass}{p-2}
\begin{sphinxuseclass}{card}
\begin{sphinxuseclass}{w-100}
\begin{sphinxuseclass}{shadow}
\begin{sphinxuseclass}{card-body}\begin{itemize}
\item {} 
\sphinxAtStartPar
Measurement

\item {} 
\sphinxAtStartPar
Set point

\item {} 
\sphinxAtStartPar
Status

\end{itemize}

\end{sphinxuseclass}
\end{sphinxuseclass}
\end{sphinxuseclass}
\end{sphinxuseclass}
\end{sphinxuseclass}
\end{sphinxuseclass}
\end{sphinxuseclass}
\end{sphinxuseclass}
\end{sphinxuseclass}
\end{sphinxuseclass}
\end{sphinxuseclass}
\end{sphinxuseclass}
\end{sphinxuseclass}
\end{sphinxuseclass}\end{sphinxadmonition}

\begin{sphinxadmonition}{important}{Important:}
\sphinxAtStartPar
In the master instance, on every slot, you can see the setpoint arriving from the instance and the measurement going back;
\sphinxincludegraphics{{MO_Slot_A}.png}
\end{sphinxadmonition}







\renewcommand{\indexname}{Index}
\printindex
\end{document}